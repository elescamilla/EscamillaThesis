\chapter{Related Work}
\label{ch:relatedwork}

\todo[Smooth it out, send for review]

This is one of few studies that looks at the representation of links \textit{to} scholarly source code in scholarly literature. Previous works have investigated the opposite: representation of links \textit{to} scholarly literature \textit{from} scholarly source code repositories. Wattanakriengkrai et al. \cite{wattanakriengkrai_github_2022} studied the extent to which scholarly papers are cited in public GitHub repositories to gain key insights into the landscape of scholarly source code production, and uncovered potential problems with long-term access, tracing, and evolution of these repositories. Färber \cite{farber-jcdl2020} analyzed data from Microsoft Academic Graph, which maps publications to their source code repositories, in order to look at the content and popularity of academic source code related to published work. Färber's work focuses on the content of the GitHub repositories referenced in scholarly publications, while our work looks at how scholarly publications link to GHPs. Other related work addresses finding scholarly source code repositories hosted in GHPs, either by looking through the content of the repository or by searching for links to scholarly literature in the repositories themselves. Hasselbring et al. \cite{hasselbring} investigated public repositories on GitHub and estimated that it contained over 5,000 repositories of specifically research software -- a similar estimation to Färber. Bhattarai et al. investigated the correlation between citations and repository interaction features and trained a classifier to predict whether a paper would be highly cited based on the interaction features of the repository included in the publication \cite{bhattarai-jcdl2022}. The study looked at a range of interaction features and found that user engagement metrics, namely forks, stars, subscriptions, and issues, are the only attributes that had statistically significant correlation with citations across the analyzed timespan. 

Understanding the extent to which scholarly articles reference source code is important because scholarly materials that are hosted on the Web are vulnerable to decay in the same manner as Web resources in general. In 2014, Klein et al. \cite{klein-plos2014} analyzed the use of URIs to the Web at large in 3.5 million scholarly articles published in arXiv, Elsevier, and PMC corpora from 1997 to 2012. They found that the number of general URIs used in scholarly publications rapidly increased from 1997 to 2012. However, they also found that reference rot affects nearly 20\% of Science, Technology, and Medicine (STM) publications. When looking specifically at publications with at least one Web reference, seven out of ten publications are affected by reference rot. Reference rot is a general term that indicates that either link rot or content drift has altered the content of the Web page to be different than the content to which the author was originally referring \cite{vandesompel-icm2014}. Link rot occurs when the URI that was originally referenced is completely inaccessible. Link rot can cause the ``404: Page not found'' error that most Web users have experienced. Content drift occurs when the content that was originally referenced by a URI is different from the content currently available at the URI. Both link rot and content drift are a result of the dynamic and ephemeral nature of the Web. In a study on the same corpus studied by Klein et al. \cite{klein-plos2014}, Jones et al. \cite{jones-plos2016} found that 75\% of references suffer from content drift. Additionally, they found that the occurrence and impact of content drift increases over time. In 2015, only 25\% of referenced resources from 2012 publications were unchanged and, worse yet, only 10\% of publications from 2006 were unchanged.

% Despite their widespread use, GHPs are not permanent and neither are the repositories that they host. Gitorious and Google Code are two examples of code hosting platforms that no longer exist. As a result, any software products hosted on these platforms that were not moved to another platform are no longer available and any URIs to either of these platforms included in publications no longer point to the resources the author intended.  

Understanding the scope of how scholarly source code is represented in scholarly literature is vital to strengthening efforts to preserve this code and make it available for the long-term, as a part of the scholarly record. Some scholars take an active role in the long-term preservation of their software and engage in a strategy known as self-archiving. Self-archiving puts the responsibility on scholars to deposit their code product into a repository that guarantees long-term preservation, like Zenodo \cite{peters_zenodo} or the Open Science Framework \cite{foster_osf}. However, a study by Milliken et al. \cite{iasge_enviro_scan} found that only 47.2\% of the academics who create software products self-archive their software. While self-archiving can help safeguard research software, it has yet to become common practice for scholars.

Presently, neither self-archived code nor programmatically captured code incorporates the scholarly ephemera that can help secondary readers understand and evaluate the source code being cited. This is where Web archiving may be beneficial. Web archiving's goal lies in preserving the Web so that users can see a Web page as it existed at a certain point in time, which is helpful for archiving source code and the accompanying scholarly ephemera. However, because of the resources it takes to archive the Web, automated Web archiving services like the Internet Archive will crawl the most visited Web pages frequently, while the least visited Web pages, including scholarly content, may never be fully captured. Although the Internet Archive includes some GHP sites, it cannot be depended upon to preserve any given page in its entirety. Other Web archiving tools like the Webrecorder suite \cite{webrecorder} provide higher quality captures of source code and ephemera, but take more time, resulting in decreased scalability for archiving the Web at large. Also, while current Web archiving implementations are well-suited for archiving the scholarly ephemera around scholarly code, they are less effective with the source code itself, which has different metadata and reuse needs than a typical Web page. 

Software Heritage preserves source code and its development history from the perspective that source code is itself a valuable form of knowledge that should be captured, including the unique evolution of the source code to create the code product at a given point in time \cite{dicosmo-ipres2017}. Software Heritage provides a central repository containing the source code and development histories of millions of code products across programming languages, hosting platforms, and package repositories. The result is a repository that researchers can use as a more representative sample than a single hosting platform or package library to analyze source code. For instance, Pietri et al. \cite{pietri-msr2020} and Bhattacharjee et al. \cite{bhattacharjee-msr2020} leverage the scope of the Software Heritage dataset to analyze trends in software development across a more heterogeneous dataset than could be found in a single hosting platform. While studies like these have made use of the holdings of Software Heritage, they do not analyze what has or has not been preserved in Software Heritage.

We know that: a) materials hosted on the Web and cited in scholarly literature are subject to reference rot, b) source code and its important scholarly ephemera are particularly at risk because of a lack of holistic archiving,
and c) source code is being cited more in our scholarly literature. To understand the scope of source code citations and quantify the risk of loss, we analyzed a corpus of scholarly publications and the URIs to GHPs that the publications contain. This study will also investigate the prevalence of URIs to data and software products and identify the other Web-based repositories and hosting services that scholars are using and citing in scholarly work. 
