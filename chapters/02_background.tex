\chapter{Background}
\label{ch:background}

\todo[Refine and beef up, add ephemera example, send for review]

Git Hosting Platforms (GHPs) are online code repository platforms that are commonly used by software developers, including researchers, to host software because they support version control, collaboration, and discoverability. GitHub, GitLab, Bitbucket, and SourceForge are examples of popular GHPs. Despite their popularity, GHPs are not permanent and neither are the repositories that they host. Gitorious and Google Code are two examples of code hosting platforms that no longer exist. As a result, any software products hosted on these platforms that were not moved to another platform are no longer available and any URIs to either of these platforms included in publications no longer point to the resources the author intended.  

GHPs are Web-based hosting services that use and extend the functionality afforded by git, a version control system. For example, GitHub offers pages for issues, pull requests, wikis, and other additional information that is outside the scope of the git version control system but adds to the development experience. These additional pages are called ephemera and add context to the development of the repository. From a Web archiving perspective, preserving repositories hosted in GHPs is focused on preserving the Web representation of the repository within GitHub and archiving the look and feel a user would have experienced at that date and time. This approach preserves the ephemera that would not be captured with a \verb|git clone| or copy of the code. 

To look at the importance of preserving ephemera, we will look at the GitHub repository for Keras\footnote{\url{https://github.com/keras-team/keras}}, a ``deep learning API written in Python'' and, by far, the most popular GitHub repository referenced in the corpora we studied. On June 14, 2023, we created three mementos for the Keras repository: the home page\footnote{URI-R: \url{https://github.com/keras-team/keras}}\footnote{URI-M: \url{https://web.archive.org/web/20230614175747/https://github.com/keras-team/keras}}, the first page of issues\footnote{URI-R: \url{https://github.com/keras-team/keras/issues}}\footnotes{URI-M: \url{https://web.archive.org/web/20230614175841/https://github.com/keras-team/keras/issues}}, and the second page of issues\footnote{URI-R: \url{https://github.com/keras-team/keras/issues?page=2\&q=is\%3Aissue+is\%3Aopen}}\footnote{URI-M: \url{https://web.archive.org/web/20230614180412/https://github.com/keras-team/keras/issues?page=2\&q=is\%3Aissue+is\%3Aopen}}. After ten days, we created three more mementos, one for each page. We used the Compare tool from the Internet Archive's Wayback Machine to compare the URI-Ms for each of the three pages. EXPLAIN MEMENTO COMPARE SCREENSHOTS. Over 10 days, X commits were made to the repository. All of those commits and code changes would be preserved by archiving the code alone. However, six new issues were created and nine issues were closed. The Issues page tells a story of the development of the code as well as the community that has created it. Archiving the Issues page and other ephemera that surround the code provides the context for the living code product and aids in our knowledge of how the code works and why certain decisions were made. 

Internet Archive and Software Heritage are two of the primary archives that contain captures of code hosted in GHPs. Intenet Archive is a Web archive and, as such, their primary goal is the preservation of the Web at large with no special emphasis on the holdings of GHPs. Web archives crawl live Web pages, or URI-Rs, and create archived versions of the live Web pages called mementos or URI-Ms. Each URI-M has an associated Memento-Datetime, the date and time that the URI-M was created. Each memento in Internet archive is uniquely identified with a combination of the URI-R and the Memento-Datetime. As a Web archive, mementos created by Internet Archive contain both the software product and the surrounding ephemera to allow users a complete picture of the hosted repository as it was available on the live Web.

Software Heritage is a non-profit organization that works to ``collect, preserve, and share all software that is publicly available in source code form'' \cite{swh-mission}. The Software Heritage (SWH) naming convention differs from the terminology defined in the Memento framework \cite{mementoprotocol}. In Software Heritage, the URI for a repository is an origin and each copy of the repository is a capture. A persistent identifier is created for each artifact within the capture, called a SWHID. While Web archives typically have a large scope covering a wide variety of content types, Software Heritage is singularly focused on the archival of source code and its development history. As a result, their captures solely archive the software product hosted in the GHP and do not archive the ephemera surrounding it. 

Scholars also deposit their code and data products in Zenodo; however, we excluded Zenodo from our study because it does not support URI searches through its Web interface or API. Users can conduct text searches or find resources through direct links, but they cannot search for a URI. 